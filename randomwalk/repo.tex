\documentclass[a4j]{jarticle}
    \usepackage[dvipdfmx]{graphicx}
    \usepackage[ top=25truemm,bottom=37truemm,left=25truemm,right=25truemm]
    {geometry}
    \usepackage{ascmac}
    \usepackage{amsmath}
    \usepackage{array}
    \usepackage{here}
    \usepackage{url}
    \usepackage{listings, jlisting}
    \usepackage{cases}
    \usepackage[subrefformat=parens]{subcaption}
    \renewcommand{\lstlistingname}{リスト}
\lstset{language=c,
  basicstyle=\ttfamily\scriptsize,
  commentstyle=\textit,
  classoffset=1,
  keywordstyle=\bfseries,
  frame=tRBl,
  framesep=5pt,
  showstringspaces=false,
  numbers=left,
  stepnumber=1,
  numberstyle=\tiny,
  tabsize=4
}


    \begin{document}
    \section{ランダムウォーク}
    $S_1,S_2,\dots S_n$,i.i.dとし,$P(S_i=1)=p,P(S_i=-1)=1-p$とする($L=1$に限定して考える). i.i.dとは独立同分布を意味する.独立同分布とは簡単に言えば,ある時間に左右どちらに移動するかは,未来や過去にどちらに移動したかによらない.
    さらに,どの時間においても,右に行く確率$p$,左に行く確率$1-p$ということである.$n \geq 1$に対して$X_n$を式(\ref{xn})のように定義する.時刻$n$とすると$X_n$はランダムウォークである.
  \begin{eqnarray}
    X_n = S_1 + S_2 + \dots + S_n \label{xn}
  \end{eqnarray}

  さて,$P(X_n=k)$ ($k$:整数,$-n \leq k \leq n$)が分かれば,分散を計算することができる. $P(X_n=k)$は$n$秒後に位置$k$にいる確率を表している.このため$P(X_n=k)$を計算する.
  $S_i=1$なるiの個数$W_n$,$S_i=-1$なるiの個数$L_n$と定義する.これより,$X_n=W_n-L_n$,$n=W_n+L_n$であるから,$X_n+n=2W_n$と表せる.
  すなわち$X_n=k$は$W_n=\frac{k+n}{2}$というふうに表せる.
  $W_n$は$n$回試行で$i$回1がでた確率だから二項分布$Bin(n,p)$に従う. これより$k+n$が奇数のとき,$\frac{k+n}{2}$は整数にならないから,$P(X_n =k)=0$となる.
  $k+n$が偶数のときは式(\ref{px})に示すように$P(X_n=k)$が計算できる.
  \begin{eqnarray}
    P(X_n=k) = P\left( W_n = \frac{n+k}{2} \right) = {}_n C _{\frac{n+k}{2}} p^{\frac{n+k}{2}} (1-p)^{\frac{n-k}{2}} \label{px}
  \end{eqnarray}

  式(\ref{px})より$X_n$の期待値および分散は式(\ref{ex})および式(\ref{vx})になる.
  \begin{eqnarray}
    E[X_n] &=& E[2W_n -n] = 2np-n = n(2p-1) \label{ex} \\
    V[X_n] &=& V[2W_n -n] = 4V[W_n] = 4np(1-p) \label{vx}
  \end{eqnarray}

  $p=0.5$のとき,期待値と分散を計算してみる.式(\ref{ex})および式(\ref{vx})に$p=0.5$を代入すると式(\ref{ex05})および式(\ref{vx05})のようになる.
  式(\ref{vx05})は,数値計算で求めた分散$V[\Delta x^2(N)]=N$になることを示している.つまり,最小二乗法の傾きはおおよそ$1.0$になるはずである.
  \begin{eqnarray}
    E[X_n] &=&  n(2p-1) = 0 \label{ex05} \\
    V[X_n] &=& 4np(1-p) = n \label{vx05}
  \end{eqnarray}

  $p=0.7$のとき,期待値と分散を計算してみる.式(\ref{ex})および式(\ref{vx})に$p=0.7$を代入すると式(\ref{ex07})および式(\ref{vx07})のようになる.
  式(\ref{vx07})は,数値計算で求めた分散$V[\Delta x^2(N)]=0.84N$になることを示している.つまり,最小二乗法の傾きはおおよそ$0.84$になるはずである.
  \begin{eqnarray}
    E[X_n] &=&  n(2p-1) = 0.4n \label{ex07} \\
    V[X_n] &=& 4np(1-p) = 0.84n \label{vx07}
  \end{eqnarray}

        \begin{thebibliography}{9}
          \bibitem{kiso}  久保田達也,"現代数理統計学の基礎",共立出版社,2020
          \bibitem{aisia} AIcia Solid Project,\url{https://youtu.be/NE1W0wJH8q8},閲覧日2021/1/21
          \end{thebibliography}
\end{document}

