\documentclass[a4j]{jarticle}
    \usepackage[dvipdfmx]{graphicx}
    \usepackage[ top=25truemm,bottom=25truemm,left=25truemm,right=25truemm]
    {geometry}
    \usepackage{ascmac}
    \usepackage{array}
    \usepackage{here}
    \usepackage{url}
    \usepackage{listings, jlisting}
    \renewcommand{\lstlistingname}{リスト}
\lstset{language=c,
  basicstyle=\ttfamily\scriptsize,
  commentstyle=\textit,
  classoffset=1,
  keywordstyle=\bfseries,
  frame=tRBl,
  framesep=5pt,
  showstringspaces=false,
  numbers=left,
  stepnumber=1,
  numberstyle=\tiny,
  tabsize=4
}

\makeatletter
\def\@thesis{シミュレーション レポート}
\def\id#1{\def\@id{#1}}
\def\department#1{\def\@department{#1}}

\def\@maketitle{
\begin{center}
{\huge \@thesis \par} %修士論文と記載される部分
\vspace{10mm}
{\LARGE\bf \@title \par}% 論文のタイトル部分
\vspace{10mm}
{\Large \@date\par}	% 提出年月日部分
\vspace{20mm}
{\Large \@department \par}	% 所属部分
{\Large 学籍番号 \@id \par}	% 学籍番号部分
\vspace{10mm}
{\Large 氏名 \@author}% 氏名 
\end{center}
\par\vskip 1.5em
}

\title{第6回~第10回}
\date{提出日 2020年7月15日 1~2コマ目}
\department{組番号 408}
\id{17406}
\author{金澤雄大}

    \begin{document}
    \maketitle
    \thispagestyle{empty}
    \clearpage
    \addtocounter{page}{-1}
    \section{目的}

    \section{実験環境}
      実験環境を表\ref{env}に示す.gccはWindows上のWSL(Windows Subsystem for Linux)で動作するものを用いる.
      \begin{table}[H]
        \caption{実験環境}
      \label{env}
      \begin{center}
          \begin{tabular}{c|l}\hline
            CPU & AMD Ryzen 5 3600 6-Core Processor \\ 
            メモリ & 16.0GB DDR4 \\
            OS & Microsoft Windows 10 Home \\
            gcc & (Ubuntu 9.3.0-17ubuntu1~20.04) 9.3.0 \\
            Make & GNU Make 4.2.1 \\ \hline
          \end{tabular}
      \end{center}
      \end{table}

      \section{課題6}
      本章では課題6における課題内容,プログラムの説明,実験結果,考察の4つについて述べる.
      \subsection{課題内容}
      

        \begin{thebibliography}{9}
          \bibitem{NNCT}  国立高専機構長野高専,\url{http://www.nagano-nct.ac.jp/} ,閲覧日2020年8月5日
          \end{thebibliography}
\end{document}

